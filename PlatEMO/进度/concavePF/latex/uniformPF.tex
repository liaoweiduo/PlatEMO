\documentclass[conference]{IEEEtran}

\IEEEoverridecommandlockouts


\makeatletter
\renewcommand\footnoterule{%
  \kern-3\p@
  \hrule\@width.4\columnwidth
  \kern2.6\p@}
\makeatother
  
\usepackage{graphicx}
\graphicspath{{fig/}}
\DeclareGraphicsExtensions{.eps,.pdf,.jpeg,.png}

\usepackage[misc]{ifsym}
\usepackage{multirow,booktabs,color,soul,threeparttable}
%\usepackage[ruled,vlined]{algorithm2e}
\usepackage[ruled,linesnumbered]{algorithm2e}
\usepackage{amsmath,bm,setspace}
\interdisplaylinepenalty=2500
\usepackage{amsfonts}
\definecolor{hl}{rgb}{0.75,0.75,0.75}
\sethlcolor{hl}
\usepackage{algorithmic}
\renewcommand{\algorithmicrequire}{ \textbf{Input:}} %Use Input in the format of Algorithm
\renewcommand{\algorithmicensure}{ \textbf{Output:}} %UseOutput in the format of Algorithm
\usepackage{caption}
\captionsetup{font={footnotesize}}
\usepackage{subfig}
\usepackage{afterpage}
\usepackage{array}
\usepackage{stfloats}
\fnbelowfloat
\usepackage{url}
\usepackage{cite}
\def\tablename{Table}

\usepackage[keeplastbox]{flushend}

\newcommand{\semitextbf}[1]{%
	\pdfliteral direct {2 Tr 0.3 w} %the second factor is the boldness
	#1%
	\pdfliteral direct {0 Tr 0 w}%
}

%\def\Plus{\texttt{+}}
\newcommand{\Plus}{\raisebox{.4\height}{\scalebox{.6}{+}}}
% correct bad hyphenation here
\hyphenation{op-tical net-works semi-conduc-tor}

\renewcommand\IEEEkeywordsname{Keywords}

\begin{document}

\title{Uniform Solution Set}

%\author{\IEEEauthorblockN{Linjun He\IEEEauthorrefmark{1},
%Auraham Camacho\IEEEauthorrefmark{2} and
%Hisao Ishibuchi\IEEEauthorrefmark{1}$\textsuperscript{(\Letter)}$}
%\IEEEauthorblockA{\IEEEauthorrefmark{1}School of Electrical and Computer Engineering\\
%Southern University of Science and Technology, Shenzhen, China\\
%Email: helj@mail.sustech.edu.cn, hisao@sustech.edu.cn}
%\and
%\IEEEauthorblockA{\IEEEauthorrefmark{2}CINVESTAV, Tamaulipas, Mexico\\
%Email: acamacho@tamps.cinvestav.mx}}

% use for special paper notices
%\IEEEspecialpapernotice{(Invited Paper)}

\author{\IEEEauthorblockN{Weiduo Liao, Hisao Ishibuchi} %$\textsuperscript{(\Letter)}$
\IEEEauthorblockA{
  Shenzhen Key Laboratory of Computational Intelligence, \\
  University Key Laboratory of Evolving Intelligent Systems of Guangdong Province, \\
  Department of Computer Science and Engineering, \\
  Southern University of Science and Technology (SUSTech), Shenzhen, China \\
  Email: 11849249@mail.sustech.edu.cn; hisao@sustech.edu.cn}
% \and
% \IEEEauthorblockN{Ke Shang}
% \IEEEauthorblockA{Department of Computer Science\\ and Engineering\\
% Southern University of Science\\ and Technology\\ Shenzhen, China\\
% Email: kshang@foxmail.com}
% \and
% \IEEEauthorblockN{Hisao Ishibuchi}
% \IEEEauthorblockA{Department of Computer Science\\ and Engineering\\
% Southern University of Science\\ and Technology\\ Shenzhen, China\\
% Email: hisao@sustech.edu.cn}
}


% make the title area
\maketitle

% As a general rule, do not put math, special symbols or citations
% in the abstract
\begin{abstract}
% -------------------------------------------------------------------------------------
% 
% -------------------------------------------------------------------------------------

\end{abstract}

\begin{IEEEkeywords}
Reference point specification; SMS-EMOA; hypervolume; evolutionary multi-objective optimization; 
convergence detection; dynamic mechanism 
\end{IEEEkeywords}
    
\let\thefootnote\relax\footnotetext{
  This work was supported by National Natural Science Foundation of China (Grant No. 61876075), 
  the Program for Guangdong Introducing Innovative and Enterpreneurial Teams (Grant No. 2017ZT07X386), 
  Shenzhen Peacock Plan (Grant No. KQTD2016112514355531), 
  the Science and Technology Innovation Committee Foundation of Shenzhen (Grant No. ZDSYS201703031748284), 
  and the Program for University Key Laboratory of Guangdong Province (Grant No. 2017KSYS008).
  Corresponding Author: Hisao Ishibuchi (hisao@sustech.edu.cn)
}
% For peer review papers, you can put extra information on the cover
% page as needed:
% \ifCLASSOPTIONpeerreview
% \begin{center} \bfseries EDICS Category: 3-BBND \end{center}
% \fi
%
% For perreview papers, this IEEEtran command inserts a page break and
% creates the second title. It will be ignored for other modes.
\IEEEpeerreviewmaketitle

% -------------------------------------------------------------------------------------
% outline:
% linear; concave; convex shape PF, each solution HV contribution different
% 
% 定理 (indicator based algorithm): 
%   最优分布solution set的每一个解的HV contribution需要一样(连续的PF)
%      这样这个solution set才能稳定,不然理论上indicator based algorithm会继续删除最小找大的
%     证明:例子是有一个solution set一个solution小,一个solution大,必然可以有一个解替换小的,移动
%          一下(前提 PF连续)
%   拓展:disjoint PF 的每个连续的sub PF中也hold上述
%   
% after mapping (normalize -> scale -> normalize) in 2D; in 3D 
% the difference decreases (after mathematical calculation)
% So if we treat these solutions using HV contribution after mapping,
%
% propose mapped HV contribution:  I(s,S) =  HVc(s',S'), S->S', s->s' 
% 
% 2个缺点 
% do not contain dominate information 
% a far away non-dominated solution intend to stay because 较小的mapped HV contribution difference 
% 
% introduce a coefficient c
% 贴图 c为0,c从0到1,c为1 3种情况的结果图, c为0和以前一样,c为1会有很远的解
% (c从0到1和c为1的结果可能相差不大!!!)
% 
% DTLZ7的结果显示对于不连续问题 此方法可以减少subPF的边缘影响(边缘密集)
% 
% -------------------------------------------------------------------------------------
\section{Introduction}
% -------------------------------------------------------------------------------------
% 
% -------------------------------------------------------------------------------------

\section{Computational Experiments}
% -------------------------------------------------------------------------------------
% 
% -------------------------------------------------------------------------------------
\subsection{Experimental Settings}
% -------------------------------------------------------------------------------------
% 
% -------------------------------------------------------------------------------------

\subsection{Computational Results}
% -------------------------------------------------------------------------------------
% 
% -------------------------------------------------------------------------------------

\section{Conclusions}
% -------------------------------------------------------------------------------------
% 
% -------------------------------------------------------------------------------------

\bibliographystyle{IEEEtran} 
\bibliography{mybibtex} 

\end{document}


